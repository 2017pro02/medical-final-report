% プロジェクト学習中間報告書書式テンプレート ver.1.0 (iso-2022-jp)

% 両面印刷する場合は `openany' を削除する
\documentclass[openany,11pt,papersize]{jsbook}

% 報告書提出用スタイルファイル
\usepackage[final]{funpro}%最終報告書
%\usepackage[middle]{funpro}%中間報告書

% 画像ファイル (EPS, EPDF, PNG) を読み込むために
\usepackage[dvipdfmx]{graphicx,color,hyperref}

% 'hyperref'パッケージのオプションを指定する.
% PDFのブックマークにセクション番号を挿入する.
\hypersetup{
    bookmarksnumbered
}

% 'hyperref'パッケージによって作成されるPDFのブックマークにおいて
% 日本語の文字化けを防ぐパッケージ
\usepackage{pxjahyper}

% ここから -->
\usepackage{calc,ifthen}
\newcounter{hoge}
\newcommand{\fake}[1]{\whiledo{\thehoge<70}{#1\stepcounter{hoge}}%
  \setcounter{hoge}{0}}
% <-- ここまで 削除してもよい

% ファイル分割のためのパッケージ
\usepackage{subfiles}

% url表記
\usepackage{url}

% 年度の指定
\thisYear{2017}

% プロジェクト名
\jProjectName{使ってもらって学ぶフィールド指向システムデザイン2017}

% [簡易版のプロジェクト名]{正式なプロジェクト名}
% 欧文のプロジェクト名が極端に長い(2行を超える)場合は、短い記述を
% 任意引数として渡す。
%\eProjectName[Making Delicious curry]{How to make delicious curry of Hakodate}
\eProjectName{Field Oriented System Design Learning by Users' Feedback}

% <プロジェクト番号>-<グループ名>
\ProjectNumber{2-C}

% グループ名
\jGroupName{医療グループ}
\eGroupName{Medical Group}

% プロジェクトリーダ
\ProjectLeader{1015061}{西谷歩}{Ayumi~Nishiya}

% グループリーダ
\GroupLeader  {1015174}{山根春貴}{Haruki~Yamane}

% メンバー数
\SumOfMembers{4}
% グループメンバ
\GroupMember  {1}{1015117}{佐藤碧}{Aoi~Sato}
\GroupMember  {2}{1015078}{佐藤礼於}{Leo~Sato}
\GroupMember  {3}{1015216}{堀沙枝香}{Saeka~Hori}
\GroupMember  {4}{1015259}{頼亜弥}{Aya~Rai}

% 指導教員
\jadvisor{伊藤恵,南部美砂子,奥野拓}
% 複数人数いる場合はカンマ(,)で区切る。カンマの前後に空白は入れない。
\eadvisor{Kei~Ito,Misako~Nambu,Taku~Okuno}

% 論文提出日
\jdate{2018年1月19日}
\edate{January~19, 2018}

\begin{document}
%
% 表紙
\maketitle

%前付け
\frontmatter

% 概要
\subfile{sections/0-overview}

\tableofcontents% 目次

\mainmatter% 本文のはじまり


\chapter{背景と目的}
\subfile{sections/1-haikei}


\chapter{活動概要}
\subfile{sections/2-katsudou}


\chapter{開発に向けた準備活動}
\subfile{sections/3-tchishiki}


\chapter{要件定義}
\subfile{sections/4-youken}


\chapter{ふーろぐ}
\subfile{sections/5-foolog}


\chapter{開発}
\subfile{sections/6-kaihatsu}


\chapter{評価}
\subfile{sections/7-hyouka}


\chapter{おわりに}
\subfile{sections/8-owari}


% 以降、付録(付属資料)であることを示す
\begin{appendix}
    \chapter{その他新規習得技術}
    \subfile{sections/z1-gizyutsu}

    \chapter{活用した講義}
    \subfile{sections/z2-kougi}

  %付録の終わり
\end{appendix}


%\backmatter

% 参考文献
\begin{thebibliography}{9}
    \bibitem{syourai} 二宮 利治, 2014, 日本における認知症の高齢者人口の将来推計に関する研究. \\ \url{https://mhlw-grants.niph.go.jp/niph/search/NIDD00.do?resrchNum=201405037A} (2017/12/17 アクセス)
    \bibitem{seikatsu} 羽生 春夫. 生活習慣病と認知症. 日老医誌, 2013.
    \bibitem{nutrition-dementia-00} Barberger-Gateau P, Letenneur L, Deschamps V, Peres K, Dartigues JF, Renaud S. Fish, meat, and risk of dementia. cohort study. Br Med J, 2002.
    \bibitem{nutrition-dementia-01} Morris MC, Evans DA, Bienias JL, Tangney  CC, Bennett DA, Wilson RS, Aggaewal N, Schneider J. Consumption of fish and n-3 fatty acids and risk of incident Alzheimer disease. Arch Neurol, 2003
    \bibitem{dementia-prevention-with-chinese-medicine} 岡原 一徳, 石田 康, 林 要人, 土屋 利紀. 認知症患者の行動・心理症状 (BPSD) に対する抑肝散長期投与の安全性および有効性の検討. Dementia Japan, 2012.
    \bibitem{rehabilitation} 金山 剛, 大平 雄一, 西田 宗幹, 永木 和載, 阪本 充弘, 窓場 勝之, 大脇 淳子. 回復期リハビリテーション病棟における在宅復帰患者の特徴. 理学療法科学, 2008.
    \bibitem{camera-system} 杉原 太郎, 藤波 努, 高塚 亮三. グループホームにおける認知症高齢者の見守りを支援するカメラシステム開発および導入に伴う問題. 社会技術研究論文集, 2010.
    \bibitem{dementia-prevention-with-some-programs} 福間 美紀, 塩飽 邦憲, 馬庭 留美. 高齢者の複合型認知症予防プログラムによる認知機能改善の効果. 日本農村医学会雑誌, 2014.
    \bibitem{medical-choice} 成本 迅, 「認知症高齢者の医療選択をサポートするシステムの開発」プロジェクト. 認知症の人の医療選択と意思決定支援. クリエイツかもがわ, 2016.
    \bibitem{dementia-prevention} 一宮 洋介. 認知症の予防には何をしたらよいか?. 順天堂医学, 2008.
    \bibitem{} 飯干 紀代. 『今日から実践 認知症の人とのコミュニケーション 感情と行動を理解するためのアプローチ』. 中央法規, 2011.
    \bibitem{} 井庭 崇・岡田 誠. 『旅のことば』. 丸善出2015.
    \bibitem{} 永田 久美子・桑野 康一・諏訪免 典子. 『認知症の人の見守り・SOS ネットワーク実例集 安心・安全に暮らせるまちを目指して』. 中央法2011.
    \bibitem{} 中島 京子. 『長いお別れ』, ハヤカワ・ミステリ文庫, 1976.
    \bibitem{} 酒井 保治郎・小宮 桂治. 『よくわかる脳の障害とケア』. 南江堂, 2013.
    \bibitem{} 鈴木 正典. 『認知症のための回想法』. 日本看護協会出版会, 2013.
    \bibitem{} 高橋 龍太郎. 『楽しくいきいき、認知症予防!』. インターメディカ, 2013.
    \bibitem{} 植田 孝一郎・鈴木 明子・大塚 俊男. 『認知症の人のための作業療法の手引き. ワールドプランニング』, 2010.
    \bibitem{dementia-nutrition} 植木 彰. アルツハイマー型痴呆と栄養 (特集 アルツハイマー型痴呆のリスクファクター). Japanese journal of geriatric psychiatry, 2005.
\end{thebibliography}

\end{document}

% プロジェクト学習中間報告書書式テンプレート ver.1.0 (iso-2022-jp)

% 両面印刷する場合は `openany' を削除する
\documentclass[openany,11pt,papersize]{jsbook}

% 報告書提出用スタイルファイル
\usepackage[final]{funpro}%最終報告書
%\usepackage[middle]{funpro}%中間報告書

% 画像ファイル (EPS, EPDF, PNG) を読み込むために
\usepackage[dvipdfmx]{graphicx,color,hyperref}

% ここから -->
\usepackage{calc,ifthen}
\newcounter{hoge}
\newcommand{\fake}[1]{\whiledo{\thehoge<70}{#1\stepcounter{hoge}}%
  \setcounter{hoge}{0}}
% <-- ここまで 削除してもよい

% ファイル分割のためのパッケージ
\usepackage{subfiles}

% url表記
\usepackage{url}

% 年度の指定
\thisYear{2017}

% プロジェクト名
\jProjectName{使ってもらって学ぶフィールド指向システムデザイン2017}

% [簡易版のプロジェクト名]{正式なプロジェクト名}
% 欧文のプロジェクト名が極端に長い(2行を超える)場合は、短い記述を
% 任意引数として渡す。
%\eProjectName[Making Delicious curry]{How to make delicious curry of Hakodate}
\eProjectName{Field Oriented System Design Learning by User's Feedback}

% <プロジェクト番号>-<グループ名>
\ProjectNumber{2-C}

% グループ名
\jGroupName{医療グループ}
\eGroupName{Medical Group}

% プロジェクトリーダ
\ProjectLeader{1015061}{西谷歩}{Ayumi~Nishiya}

% グループリーダ
\GroupLeader  {1015174}{山根春貴}{Haruki~Yamane}

% メンバー数
\SumOfMembers{4}
% グループメンバ
\GroupMember  {1}{1015117}{佐藤碧}{Aoi~Sato}
\GroupMember  {2}{1015078}{佐藤礼於}{Leo~Sato}
\GroupMember  {3}{1015216}{堀沙枝香}{Saeka~Hori}
\GroupMember  {4}{1015259}{頼亜弥}{Aya~Rai}

% 指導教員
\jadvisor{伊藤恵,南部美砂子,奥野拓}
% 複数人数いる場合はカンマ(,)で区切る。カンマの前後に空白は入れない。
\eadvisor{Kei~Ito,Misako~Nambu,Taku~Okuno}

% 論文提出日
\jdate{2018年1月26日}
\edate{January~26, 2018}

\begin{document}
%
% 表紙
\maketitle

%前付け
\frontmatter

% 概要
\subfile{sections/0-overview}

\tableofcontents% 目次

\mainmatter% 本文のはじまり


\chapter{背景と目的}
\subfile{sections/1-haikei}


\chapter{目的と活動概要}
\subfile{sections/2-katsudou}


\chapter{開発に向けた準備活動}
\subfile{sections/3-tchishiki}


\chapter{要件定義}
\subfile{sections/4-youken}


\chapter{ふーろぐ}
\subfile{sections/5-foolog}


\chapter{開発}
\subfile{sections/6-kaihatsu}


\chapter{評価}
\subfile{sections/7-hyouka}


% 以降、付録(付属資料)であることを示す
\begin{appendix}
  \chapter{その他新規習得技術}
  \subfile{sections/z1-gizyutsu}

  \chapter{活用した講義}
  \subfile{sections/z2-kougi}

  %付録の終わり
\end{appendix}


%\backmatter

% 参考文献
\begin{thebibliography}{9}
    \bibitem{syourai} 日本における認知症の高齢者人口の将来推計に関する研究. 二宮 利治, 2014. \\ \url{https://mhlw-grants.niph.go.jp/niph/search/NIDD00.do?resrchNum=201405037A} (2017/12/17 アクセス)
    \bibitem{seikatsu} 羽生 春夫. 生活習慣病と認知症. 日老医誌, 2013.
    \bibitem{nutrition-dementia-00} Barberger-Gateau P, Letenneur L, Deschamps V, Peres K, Dartigues JF, Renaud S. Fish, meat, and risk of dementia. cohort study. Br Med J, 2002.
    \bibitem{nutrition-dementia-01} Morris MC, Evans DA, Bienias JL, Tangney  CC, Bennett DA, Wilson RS, Aggaewal N, Schneider J. Consumption of fish and n-3 fatty acids and risk of incident Alzheimer disease. Arch Neurol, 2003
\end{thebibliography}

\end{document}

\documentclass[../report]{subfiles}
\setcounter{section}{0}
\begin{document}

本章では,システム開発に向けた準備活動について説明する.
\bunseki{佐藤碧}


\section{知識習得}
\subsection{書籍・論文} \label{sec:document-article}
プロジェクト開始時点で本グループのメンバーは,認知症についての知識を殆ど持っていない状態であった.
認知症をテーマにして課題解決に取り組む上では,認知症についてどのような問題が存在しているかを把握しなければ,解決するべき課題も見つけることができない.
そこでまずは,認知症についての知識を身につけるところから活動を始めた.
しかし,プロジェクト学習という限られた時間のなかでは,知識の習得ばかりに時間をかけることはできない.
そのため,効率よく学習を進めるべく,本グループのメンバーそれぞれが手分けをして認知症に関する知識を習得し,それをメンバー全員で共有するという手法を用いた.
なお,メンバーそれぞれの知識習得は,書籍やインターネット上に公開されている論文の閲覧するという形で行った.
また,知識の共有は,メンバーそれぞれが習得した知識をスライドやドキュメントにまとめ,要点を発表するという形で行った.
この手法によって本グループのメンバーは,認知症とはどのような病気であるのかや,その症状や種類,治療法といった基本的な知識だけでなく,現状の認知症に関する取り組みや社会問題,研究といったようなことを含めた幅広い知識を,体系的に習得していくことができた.
その際にメンバーが読んだ書籍については,参考文献の\cite{approach}から\cite{world-planning}に示した.
\bunseki{佐藤碧}

\subsection{認知症サポーター養成講座}
函館認知症の人を支える会の主催者の方々を講師として大学にお招きし,認知症サポーター養成講座を開設していただいた.
認知症サポーターとは,「認知症を正しく理解して,認知症の人や家族を温かく見守る応援者」のことである.
講座では,認知症について理解するために,認知症とは何か,認知症を引き起こす原因,中核症状と行動・心理症状といったことについて説明していただいた.
そのほかには,認知症の人の支援の仕方や出会った際の接し方・心がけなどといったことも含まれていた.
それらの内容については,概ね事前に\ref{sec:document-article}で学んだものと同様であった.
そのことから,自分たちが学んできた知識が間違いではないということを確認できた.
また単純な説明だけではなく,要所要所で実例や・講師の方たちの実体験を交えての説明をしていただくことができた.
これによって,それまでの文献から得た知識だけではなく,実際に認知症の人と接している方々ならではの切実な思いや,介護負担の大きさについて知ることができた.
認知症である方と交流できたわけではないが,座学だけでなく実際に認知症の方と交流のある方々の思いや考え方に触れることができた.
このことは,今後認知症をテーマにして課題解決に取り組んでいく上で,貴重かつ重要な経験となった.
\bunseki{佐藤碧}

\subsection{ロボット開発ワークショップ}
京都の同志社女子大学京田辺キャンパスで開催された,ロボット開発ワークショップに参加した.
このワークショップでは,レクリエーションを通して人間とのコミュニケーションを行うロボットに必要な要素の模索を行った.
具体的には,まず数人でグループを作って,そのなかで自身にとっての価値にまつわる経験を共有した.
その後,他者とのコミュニケーションの中でお互いの価値や本質を見つけ出すためのシナリオ作成をした.
つまり,人間同士がどのような行動や会話を通して,お互いの最も根幹な部分の存在に気づくかを明らかにしようということである.

今回のワークショップでは,コミュニケーションにおける会話,つまりは言葉に着目していた.
しかし本来コミュニケーションにはそれ以外の要素であるところの,非言語的な要素を多分に含んでいる.
両者は常に互いに影響を与え合いながら,コミュニケーションを成り立たせている.
つまりは不可分な存在ということである.
したがって今後は,非言語的なアプローチについても研究の余地があるといえる.

今回のワークショップ自体は,認知症との直接的な関係があるとはいえない.
しかし,京都府立医科大学の成本医師からは,本人の生活パターンや好みを基に認知症患者本人の意思を判断していることがあるというご意見を頂いていた.
よって今回のワークショップのように,人にとっての価値や本質を見つけ出そうとする試みは認知症治療における意思決定支援のシステムを構築する上で重要であると考えられる.
なぜなら,それを明らかにできれば,医療選択時にどれが本人の意志に沿った選択肢であるかということを推測する際の貴重な判断材料と成り得るからである.

認知症治療や介護において,コンピュータが発達してきた現代でもまだまだ人の手で行わなければならないことが多い.
それは,認知症の人の周囲環境や家族との人間関係などといった複雑でデリケートな問題について,パターン化することが非常に難しいからである.
そういったパターン化できない,或いはしにくい問題をコンピュータに扱わせることもまた非常に難しい.
今回のワークショップは,そういった問題を解決するためのアプローチの一つとして大変意義のあるものであったといえる.
人と人とのコミュニケーションにおいて重要になってくる,人の中核部分となるような価値や本質に迫ることができた.
その体験ができたことは単に認知症関係のシステムを開発するだけでなく,コミュニケーションについてよく考えるという意味でも貴重な経験であったといえる.
\bunseki{佐藤碧}


\section{技術習得}
\subsection{動体検出カメラ}
本システムでは,料理写真の自動撮影を行うためにRaspberry Piを用いる.
Raspberry Piはシングルボードコンピュータの一種である.
同じくシングルボードコンピュータであるArduinoと比べると,LinuxをベースにしたOSが搭載されているため,開発をスムーズに進めることができるという利点がある.
しかし,料理写真の自動撮影を担当するメンバーは今回初めてRaspberry Piを使用するため,まずは開発環境を整える傍らでRaspberry Piの使い方について学習する必要があった.
はじめに,Raspberry Piに対して自分のPCからSSHを使って接続できるようにした.
その後,開発に必要なパッケージをRaspberry Piにインストールするために,自分のPCからブリッジさせてRaspberry Piをインターネットの接続できるようにした.
そして,必要なパッケージをOSのパッケージマネージャーなどを利用してインストールし,開発環境を整えた.
インストールしたパッケージは以下のとおりである.
\begin{description}
    \item[Motion] \mbox{}\\
    接続されたカメラを介して動体を検知し,写真や動画を取ることができる.
    \item[pyenv] \mbox{}\\
    Pythonの複数のバージョン共存させたり,環境ごとにバージョンを変更したりすることができるPythonの環境分離ツールである.
    \item[OpenCV] \mbox{}\\
    オープンソースで開発されている画像処理ライブラリである.OSのパッケージマネージャーでインストールするとバージョンが古いため,ソースコードからビルドしてインストールした.
\end{description}
\bunseki{佐藤碧}


\subsection{Webアプリケーション} \label{sec:3-webapp}
本システムの案が出た頃より,ユーザ認証や高齢者と家族の関係のモデル化などが必要であることが分かっており,これを実装するのに既存のWebフレームワークを使うことで開発期間を短縮できると考えていた.
そこで夏季休暇中にRuby,及びRuby on Railsを学習し,後期からの開発に備えた.
学習の際に使用したWebサイトはRuby on Railsガイド(\url{https://railsguides.jp/}),Ruby on Rails API(\url{http://api.rubyonrails.org/})である.

また,ユーザのテレビに過去の料理画像などを表示する方法についても検討した.
地上波などで実際に流れている放送の上にテロップなどを表示する手法が本学で研究されているが,本システムでは一画面全体を利用してより多くの情報を表示する必要があるため,テロップ表示は断念した.
本グループはRaspberryPiを用意し,これをテレビにHDMI接続することで,テレビに食事の情報を表示することとした.
情報表示はWebサーバ側でテレビの画面を用意し,これをテレビに映すこととした.
テレビにリアルタイムで情報を表示するため,Ruby on RailsでWebSocketを使用する方法を習得した.

本システムを実際に高齢者の方々に使って頂くために,外部のサーバにデプロイする必要があった.
そこでHerokuというPaaSのホスティングサービスを使用して,これを実現することとした.
数あるホスティングサービスの中でHerokuを選択した理由は,本システムはPostgreSQLやRedisなどのデータベースや,ActionCableを使ったWebSocketの技術を使用しており,これらに柔軟に対応している点である.
またGitHubから簡単に導入できるため,より開発に時間を割けると考えた.
Herokuも夏季休暇中に使用方法を学習した.
\bunseki{佐藤礼於}


\subsection{料理画像認識API} \label{sec:3-food-recognition}
日々の料理画像から食習慣の偏りを検知する上で,画像から料理名を識別することは必要不可欠であった.
一から機械学習すると,サンプル画像の収集や学習にかかる時間,費用などが多くかかり,本グループの進行に影響が出ると考えられる.
よって既存の画像認識のWebAPIを使用することとした.
画像から料理名を出すことのできるWebAPIは,「Google Cloud Vision API」,「IBM Visual Recognition」,「docomo 画像認識API」などが挙げられる.
このうち,日本独自の料理に対しても有効な認識をするAPIは「docomo 画像認識API」だけであったため,料理画像認識にはこれを用いることとした.

なお,「docomo 画像認識API」の仕様上,一つの画像に一つの物体の認識に特化しており,複数の料理を入力として与えた場合に対応していない.
よって「docomo 画像認識API」に画像を与える前に皿ごとの画像に分ける必要があった.
そこでオープンソースの画像処理ライブラリであるOpenCVを使用した,WebAPIサーバを構築することとした.
foologのWebサーバとは別のサーバとした理由は,WebサーバはRubyのHeroku環境を使用しているが,OpenCVには公式でRubyをサポートしていないからである.
Rubyと使い勝手が似ているPythonを使用して,WebAPIサーバを作成する.
\bunseki{佐藤礼於}

\end{document}

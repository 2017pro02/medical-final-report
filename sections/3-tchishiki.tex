\documentclass[../report]{subfiles}
\setcounter{section}{0}
\begin{document}

本章では,認知症に関する知識習得について説明する.
\bunseki{佐藤碧}


\section{知識習得}
\bunseki{佐藤碧}


\section{技術習得}
\subsection{カメラ勉強}
\bunseki{佐藤碧}


\subsection{Web勉強}
本システムの案が出た頃より,ユーザー認証や高齢者と家族の関係のモデル化などが必要であることが分かっており,これを実装するのに既存のWebフレームワークを使うことで開発期間を短縮できると考えていた.
そこで夏季休暇中にRuby,及びRuby on Railsを学習し,後期からの開発に備えた.
学習の際に使用したウェブサイトはRuby on Railsガイド(\url{https://railsguides.jp/}),Ruby on Rails API(\url{http://api.rubyonrails.org/})である.

また,ユーザのテレビに過去の料理画像などを表示する方法についても検討した.
地上波などで実際に流れている放送の上にテロップなどを表示する手法が本学で研究されているが,本システムでは一画面全体を利用してより多くの情報を表示する必要があるため,テロップ表示は断念した.
私達はRaspberryPiを用意し,これをテレビにHDMI接続することで,テレビに食事の情報を表示することとした.
情報表示はWebサーバ側でテレビの画面を用意し,これをテレビに映すこととした.
テレビにリアルタイムで情報を表示するため,Ruby on RailsでWebSocketを使用する方法を習得した.
\bunseki{佐藤礼於}


\subsection{料理画像認識の勉強}
日々の料理画像から食習慣の偏りを検知する上で,画像から料理名を識別することは必要不可欠であった.
一から機械学習すると,サンプル画像の収集や学習にかかる時間,金銭などが考えられ,本プロジェクトの進行に影響が出ると考えられる.
よって既存の画像認識のWebAPIを使用することとした.
画像から料理名を出すことのできるWebAPIは「Google Cloud Vision API」,「IBM Visual Recognition」,「docomo 画像認識API」などが挙げられる.
このうち,日本独自の料理に対しても有効な応答をするAPIは「docomo 画像認識API」だけであったため,料理画像認識にはこれを用いることとした.
\bunseki{佐藤礼於}

\end{document}

\documentclass[../report]{subfiles}
\setcounter{section}{0}
\begin{document}

本章では,ユーザからいただいた評価について説明する.
\bunseki{頼亜弥}

\section{ユーザの評価} \label{sec:7_hyouka_monowasurecafe}
\subsection{函館認知症の人を支える会の評価}
2017年10月21日,函館市総合福祉センターで函館認知症の人を支える会の人と話し合いをした.
具体的には,「ふーろぐ」についての説明を行った後,函館認知症の人を支える会が主催している物忘れカフェでシステム体験会を開催するという提案をした.
函館認知症の人を支える会の人から頂いたレビューの一部を以下に紹介する.


\begin{itemize}
    \item 食べるということが一番大事です。だからおろそかになると、うつ病の原因にもなるので、食事に着目したという発想は良い。食生活を維持することは最後まで大事だと思う。
    \item 毎日きちっと食べるわけじゃないので,1週間で正しい食事ができているかで判断すべき.嫌いなものは除外するといった機能があると良いかも.できなければ食べるべき食事は複数提示するとか.
    \item このシステムを施設とかで使用してもらうと実用性があるかも.どんな食べ物を施設では用意されていてそれを食べているのか,撮影してそれをウェブで家族が把握できたりするのも良さそう.
    \item ただ認知症予防にはならない.20代半ばから始まり80代になって発症するものなので,もっと対象を広めるべき.
\end{itemize}
\bunseki{頼亜弥}

\subsection{高齢者の評価}
2017年11月18日,函館市総合福祉センターで行われた物忘れカフェに参加した.
このイベントでは,物忘れカフェの参加者に向けた「ふーろぐ」のシステム体験会を行った.
システム体験会では,システムの説明として小芝居をした後,参加者に実際にシステムを体験していただいた.
物忘れカフェの参加者の人から頂いたレビューの一部を以下に紹介する.

\begin{enumerate}
    \item[] ボックスについて
\begin{itemize}
    \item 目に見える機器をいじったり分解したりする人もいるので, ラズパイのような機器はしっかり隠したほうが良い.
    \item やっぱりボックスだと置く場所に困る,圧迫感がある.
    \item スタンド型が良い.
\end{itemize}

    \item[] カメラについて
\begin{itemize}
    \item 写真の自動撮影だけじゃなくて,電源についても自動でOn/Offが切り替わるようにして欲しい.
    \item ユーザに必要な操作数は多くても2つ以内に収めてほしい.
    \item 音声にするのは良いと思う.音量はTVとか結構音大きくして見てることが多いかも.音は大きめで良いと思う.
\end{itemize}

    \item[] TV画面について
\begin{itemize}
    \item おすすめの料理を表示するときに, その人の運動量や身長, 体重などにあったものを表示してほしい.
    \item コメント機能について,他の人にも共有できているということが分かり, 寂しくなくていいね.
    \item いつもの料理にコレをプラスすると良いなど,料理をすすめるだけでなく料理の傾向から何か調味料や材料のオススメがあると良い.
\end{itemize}

    \item[] Web画面について
\begin{itemize}
    \item 撮影者の安全・健康を知るために,撮影者がどのくらい食べているのかという情報がほしい.
    \item 写真見られるのはいいけど,実際に食べたのかはわからない.
\end{itemize}
\end{enumerate}
\bunseki{頼亜弥}

\subsection{医師の評価}
2017年12月4日,京都府立医科大学の成本医師から進捗報告のリクエストをいただき,本グループの進捗について報告した.
成本医師から頂いたレビューの一部を以下に紹介する.

\begin{itemize}
    \item 患者さんの活動をさりげなくモニターして生活機能の測定し,意思決定能力評価に役立つ資料を提供してもらえるのではないかと思います.
    \item このシステムの対象は「患者さん」ではないのですが,ふだんからライフログをとっておくことや,そのデータをモニタリングしておくことが,のちに様々なかたちで(例えば認知症が始まってからの機能評価などに)役に立つ.
\end{itemize}
\bunseki{頼亜弥}

\subsection{最終成果発表会での評価}
2017年12月8日,公立はこだて未来大学で行われた最終成果発表会に参加した.
この発表会では,本グループが開発したシステムと活動内容について発表した.
本グループでは,ポスター発表とデモに10分と質疑応答に5分の計15分という流れで発表した.アンケートを行った結果,発表技術は平均6.3点,発表内容は平均7.7点という結果となった.
学生や教員から頂いたレビューの一部を以下に紹介する.

\begin{enumerate}
    \item[] 発表技術について
\begin{itemize}
    \item 少し声の小さい人もいましたが聞こえてたので良かったです.
    \item 一部声が聞こえづらかった.システムに不具合があったのは残念だったが,分かりやすい説明だった.
    \item 全体的にわかりやすいプレゼンでしたが,デモの時に少し聞き取りにくい場面がありました.
\end{itemize}

    \item[] 発表内容について
\begin{itemize}
    \item システムがかなり作りこまれていて素晴らしいと思った.画像認識はイマイチだが,もう少しだと思う.
    \item 実際に高齢者の人からデータを収集していて,下調べをしっかりしていることが分かった.高齢者の人に体験してもらってどのような改良点が出たのか説明あるといいなと思った.
\end{itemize}
\end{enumerate}
\bunseki{頼亜弥}

\section{評価のまとめ}
本グループはユーザの評価をもとに,システムの改善を図った.具体的に,システムや展望などに反映した部分を以下に示す.

\begin{itemize}
    \item 1週間で正しい食事ができているかで判断できるように,栄養素をチャート表示したものを閲覧できるようにした.
    \item 「ふーろぐ」を施設で使用してもらうと実用性があるという意見を展望に反映した.
    \item 目に見える機器をいじったり分解したりする人もいるので, ラズパイのような機器をしっかり隠すようにボックスを設計した.
    \item ボックスの形状をユーザの自宅に置きやすいものに変えるという展望を立てた
    \item 電源を自動でOn/Offが切り替わるようにするという展望を立てた.
    \item ユーザ(撮影者)に必要な操作を「料理をボックスの中に入れる・取り出すこと」と「テレビのチャンネルを切り替えること」のみにした.
    \item おすすめの料理を表示するときに, その人の運動量や身長, 体重などにあったものを表示してほしいという意見を展望に反映した.
    \item いつもの料理にコレをプラスすると良いなど,料理をすすめるだけでなく料理の傾向から何か調味料や材料のオススメがあると良いという意見を展望に反映した.
\end{itemize}
\bunseki{頼亜弥}

\end{document}

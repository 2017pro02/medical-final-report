\documentclass[../report]{subfiles}
\begin{document}

\begin{description}
    \item[Ruby on Rails] \mbox{} \\
        foologのウェブサーバを開発するために使用した.
        Rubyで実装されたMVCアーキテクチャのWebフレームワークである.
        他のWebフレームワークと比較すると少ないコードで実装が可能となっているため,プログラミング初心者でも扱える点が特徴である.
        またRails組み込みのライブラリが多くあり,これらを用いることでより多くの機能を簡単に実装することができる.

    \item[OpenCV] \mbox{} \\
        食事の皿を認識するために使用した.
        インテルが開発・公開した,C++やPythonなどで使用できる画像処理・解析ライブラリである.
        色調変更や輪郭抽出などの画像処理を行うことができる.
        また機械学習などのAPIも公開されているため,幅広い用途で使用することができる.

    \item[Docker] \mbox{} \\
        開発環境を統一し,また開発環境の構築の手間を減らすために使用した.
        Linuxコンテナ技術を使用してコンテナ型の仮想環境を構築するソフトウェアである.
        ハイパーバイザー型の仮想化ソフトウェアと比較して,ディスク容量が少なく,インスタンスの作成・起動が速い,性能劣化がほとんどないという利点を持つ.

    \item[Heroku] \mbox{} \\
        ウェブサーバやAPIサーバをデプロイするために使用した.
        2007年にアメリカ合衆国で創業したHerokuの,PaaS型のホスティングサービスである.
        ハードウェアからアプリケーションまでの管理をHerokuが行うため,サービス稼働にかかる手間を大きく減らすことができる.
        またGitHubのリポジトリのmasterブランチにプッシュすると自動的にデプロイを行うなどといった一連の動作を設定できる.
\end{description}
\bunseki{佐藤礼於}

\end{document}
